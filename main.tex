\documentclass{article}
\usepackage[utf8]{inputenc}

\title{Deep Learning Classification Project MGI Vlasakker Zurcher}
\author{Raphael Zürcher \& Robert van de Vlasakker}
\date{April 2021}

\usepackage{natbib}
\usepackage{graphicx}
\graphicspath{{figures/}}

\begin{document}

\maketitle

\section{Introduction}
- compare resnet34 with own model
- compare reduction none (sum; mean) > overrepresented classes 
- why adapted resnet
- Sigmoid --> because we need multiple classes output
- Adapted resnet to have 17 output classes
- Talk about the class distribution
- adapt epochs > time consumption



\section{Methods}
\subsection{Data Class}
The data set contain 2100 RGB images with a resolution of 256 by 256 and a label file with the image name and their corresponding labels.
The data set consists of a CVS file and a folder containing different folder (named after their class) containing the images.
We created a Data Class for the image data set using Pytorch. 
The Data Class takes to input variables: the image location folder and the CVS file contain the image labels.
Inside the data class the image location folder is used to read all the image locations.
These locations are all saved in a list and the list is sorted alphabetically so that is matched the CSV label list.
The CVS file is opened as a Pandas data frame and the list containing the image locations is added to the data frame as a new column.
Next, both the image and their corresponding label are loaded as a tensor.
The image is first transformed to a 256 by 256 tensor just in case. 
Finally, the Data Class returns both tensors.

\subsection{Loading the Data}
Because the data is loaded as a custom Pytorch dataset splitting the data into a train set and a test set can be done with the utility tool random\_split().
We used 80\% for training (1680 samples) and 20\% for testing (420 samples).
While it is possible to calculate the optimal batch size with \(Max Batch = GPU memory  / 4 / (Tensor Size + Trainable Parameters)\), we use an arbitrary value of 16.
We are working on Google Colab and we do not know how much GPU memory is located to our Jupyter Notebook and we train different model with different trainable parameters.
A batch size of 16 is also used because to the power of 2 generally works better \citep{DBLP:journals/corr/KeskarMNST16}.
The result of loading a single batch is a tensor of [16, 3, 256, 256].

\subsection{Models}
We use two models in this project; a pretrained ResNet34 model \citep{he2016deep} with adapted linear layer and output layer and a new build from scratch model.

\section{Results}
\section{Discussion}
\section{Conclusion}



\bibliographystyle{plain}
\bibliography{references}

\end{document}